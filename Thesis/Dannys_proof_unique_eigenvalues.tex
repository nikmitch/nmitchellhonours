\documentclass[a4paper,10pt]{article}
\usepackage[utf8]{inputenc}

%opening
\title{}
\author{}

\begin{document}

\maketitle

\begin{abstract}

\end{abstract}

\section{Danny's proof of unique eigenvalues}
However, before
we attempt to disprove the existence of the possibility of a common irrational
unit of energy, we prove an auxiliary lemma.

\begin{lemma}[Distinct energy eigenvalues]
    The elements of the sequence of energy eigenvalues
    \begin{equation*}
        E_{k} = \cos{\!\left ( \frac{k \pi}{N+1} \right )}
        \qquad k=1, 2, \dots N,
    \end{equation*}
    are distinct.
\end{lemma}
\begin{proof}
    Let us prove this statement via reductio ad absurdum. Thus assume that there
    are distinct integers, $k_{1}$ and $k_{2}$, between 1 and $N$ inclusively,
    such that the equality $E_{k_{1}} = E_{k_{2}}$ is satisfied. Therefore
    \begin{align*}
        E_{k_{1}}
        &=
        E_{k_{2}}
        \\
        \cos{\!\left ( \frac{k_{1} \pi}{N+1} \right )}
        &=
        \cos{\!\left ( \frac{k_{2} \pi}{N+1} \right )}
    \end{align*}
    Rearranging the equation on one side and using the trigonometric identity
    \begin{equation*}
        \cos{\!(\alpha)} - \cos{\!(\beta)}
        =
        -2 \sin{\!\left ( \frac{\alpha+\beta}{2} \right )}
        \sin{\!\left ( \frac{\alpha - \beta}{2} \right )}
    \end{equation*}
    for the difference of two cosine functions we obtain
    \begin{equation*}
        0
        =
        E_{k_{1}} - E_{k_{2}}
        =
        -2
        \sin{\!\left ( \frac{k_{1} + k_{2}}{2(N+1)} \,\pi\right )}
        \sin{\!\left ( \frac{k_{1} - k_{2}}{2(N+1)} \,\pi\right )}.
    \end{equation*}
    A product vanishes only if at least one of its factors vanishes, thus we
    have to consider two cases:
    \begin{equation*}
        \sin{\!\left ( \frac{k_{1} + k_{2}}{2(N+1)} \,\pi\right )} = 0
        \qquad \text{and/or} \qquad
        \sin{\!\left ( \frac{k_{1} - k_{2}}{2(N+1)} \,\pi\right )} = 0.
    \end{equation*}
    Since $k_{1}$ and $k_{2}$ are distinct and their minimal value is unity
    while their maximal value is $N$, the sum $k_{1}+k_{2}$ is an integer
    between $3$ and $2N-1$. Consequently the argument of the first sine function
    cannot ever be an integer multiple of $\pi$. Similarly the difference $k_{1}
    - k_{2}$ varies between $1-N$ and $N-1$, but cannot take the value $0$ due
    to our assumption that $k_{1}$ and $k_{2}$ are distinct. Therefore the ratio
    $(k_{1}-k_{2})/(2(N-1))$ is never an integer.

    In summary we found that the equation $E_{k_{1}} = E_{k_{2}}$ does not
    have an integer solution $(k_{1}, k_{2})$ in the allowed set of integers
    between 1 and $N$. We arrived at a contradiction, since we assumed that
    there is a solution. We thus have to conclude the energy eigenvalues in the
    set $\lbrace E_{k} \rbrace$ are all distinct.
\end{proof}

\end{document}
