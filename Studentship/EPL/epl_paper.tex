\documentclass[doublecol]{epl2}
\usepackage{todonotes}
% or \documentclass[page-classic]{epl2} for one column style

\title{Thermalisation of interacting bosons on a lattice}
\shorttitle{Title} %Insert here a short version of the title if it exceeds 70 characters

\author{N. Mitchell\inst{1} \and 
D. Hutchinson\inst{1} \and 
D. Schumayer\inst{1}}

\institute{                    
  \inst{1} University of Otago - Address\\
  
}
\pacs{nn.mm.xx}{First pacs description}
\pacs{nn.mm.xx}{Second pacs description}
\pacs{nn.mm.xx}{Third pacs description}
%what are pacs?

\abstract{This paper is inspired by work done by Marcos Rigol 
et al. \ref{Rigol2007}, who showed numerically that a one-dimensional gas of 
hard-core bosons on an optical lattice (an integrable system) relaxes to an
equilibrium state that is characterised by the Generalised Gibbs Ensemble (GGE).
The GGE is a thermodynamic ensemble that has been extended to take into account
all of the conserved quantities of a system. We explore the question of whether
similar behaviour can be seen in smaller systems, which we can solve more 
exactly, where the strength of the interactions is increased to approach the 
hard-core limit.
}


\begin{document}

\maketitle


\section{Section title}
\todo{should the institute be the DWC or the uni?}
We consider a gas of $P$ bosons initially in the ground state of a box with 
$N_{in}=P$ lattice sites, and allow it to expand into a box with $N$ lattice sites.
This is similar to the setup in \ref{Rigol2007}, though we shall be using 
smaller system sizes.
We consider only repulsive interactions, so the ground state of the initial 
system corresponds to single \todo{singular?}occupation of each lattice site. 


The Hamiltonian that describes our system is 
\begin{equation*}
    \hat{H}
    =
    J \sum_{i}{(\hat{a}^\dagger_{i}\hat{a}_{i+1} + \text{h.c.})} +
    U\sum_{i}{\hat{a}^\dagger_{i} \hat{a}^\dagger_{i} \hat{a}_{i}\hat{a}_{i}}.
\end{equation*},
where [explain meaning of operators etc]

Our Hamiltonian is different from the one used by Rigol et al. in that we 
explicitly include the interaction strength. We tune this variable to 
investigate whether our system behaves like the one that they used
in the strongly interacting (hard-core) limit.\todo{can I say this, or is it
wrong because hard-core is technically infinitely repulsive, not just strong?}

The main observable of interest is the quasimomentum $\langle\hat{f}(k)\rangle$,
given by (insert equation here).

We consider the height of the distrubtion at $k=0$ over time and how this 
changes as we tune the interaction strength $U$. We also consider the shape of
the entire quasimomentum distribution for varying interaction strengths in 
the long-time limit.



See fig.~\ref{fig.1}, table~\ref{tab.1} and eq.~(\ref{eq.1}).
See also~\cite{b.a,b.b}.
\begin{equation}
\label{eq.1}
0\neq1
\end{equation}


\begin{figure}
\onefigure{epl-template.eps}
\caption{Figure caption.}
\label{fig.1}
\end{figure}


\begin{table}
\caption{Table caption.}
\label{tab.1}
\begin{center}
\begin{tabular}{lcr}
first  & table & row\\
second & table & row
\end{tabular}
\end{center}
\end{table}



\acknowledgments
Insert here the text.

\begin{thebibliography}{0}

\bibitem{b.a}
  \Name{Author F., Author S. \and Author T.}
  \REVIEW{Some Rev. A}{69}{1969}{9691}.

\bibitem{b.b}
  \Name{Author F. \and Author S.}
  \Book{Some Book of Interest}
  \Editor{A. Editor}
  \Vol{9}
  \Publ{Publishing house, City}
  \Year{1939}
  \Page{666}.

\bibitem{b.c}
  \Editor{Editor A.}
  \Book{Some Book of Interest}
  \Vol{9}
  \Publ{Publishing house, City}
  \Year{1939}
  \Section{A}.

\end{thebibliography}

\end{document}

