\documentclass[a4paper,10pt]{article}
\usepackage[utf8]{inputenc}
\usepackage{graphicx}
\usepackage{gnuplottex}
\usepackage{gnuplot-lua-tikz}
\usepackage{float}
\usepackage{listings}
\usepackage{braket}
\usepackage{mathtools}
\usepackage[mathscr]{euscript}
\usepackage{amsmath, amsfonts, amssymb, amsbsy, amsthm}


\begin{document}
\begin{abstract}
This document will summarise what my results are now that I have altered the 
code in an attempt to fix symmetry issues.
\end{abstract}

\section{Noninteracting cases}
The noninteracting cases now appear to be identical for the GPE code and the 
many-body code. 


\begin{figure}[H]
   \centering
   \begin{gnuplot}[terminal=cairolatex, terminaloptions={lw 2}, scale=0.95]
       set yrange [0:1];
       set xrange [0:50];
       set xlabel "$\\displaystyle \\frac{t}{J}$"
       set ylabel "$\\langle n_{1} \\rangle$"

       plot "./Data_for_comparisons/3by1_U0_T50_MB.dat"  u 1:4 w l lc 1 t "", \
            "./Data_for_comparisons/U0_RKF_3by1_T50.dat" u 1:2 w l lc 2 t ""        
    \end{gnuplot}
    \vspace*{-5mm}
    \caption{Similarity of output of many-body and GPE code for a $3\times1$
    lattice with no interactions.}
\end{figure}
Note that the many-body graph no longer has ``bumps'' near its minima.


\begin{figure}[H]
   \centering
   \begin{gnuplot}[terminal=cairolatex, terminaloptions={lw 2}, scale=0.95]
       set yrange [0:1];
       set xrange [0:50];
       set xlabel "$\\displaystyle \\frac{t}{J}$"
       set ylabel "$\\langle n_{1} \\rangle$"

       plot "./Data_for_comparisons/4by1_U0_T50_MB.dat"  u 1:5 w l lc 1 t "", \
            "./Data_for_comparisons/U0_RKF_4by1_T50.dat" u 1:2 w l lc 2 t ""        
    \end{gnuplot}
    \vspace*{-5mm}
    \caption{Similarity of output of many-body and GPE code for a $4\times1$
    lattice with no interactions.}
\end{figure}

The similarity continues when we look at $2$ dimensional noninteracting systems

\begin{figure}[H]
   \centering
   \begin{gnuplot}[terminal=cairolatex, terminaloptions={lw 2}, scale=0.95]
       set yrange [0:1];
       set xrange [0:50];
       set xlabel "$\\displaystyle \\frac{t}{J}$"
       set ylabel "$\\langle n_{1} \\rangle$"

       plot "./Data_for_comparisons/3by3_U0_T500_MB.dat"  u 1:10 w l lc 1 t "", \
            "./Data_for_comparisons/U0_RKF_3by3_T500.dat" u 1:2 w l lc 2 t ""        
    \end{gnuplot}
    \vspace*{-5mm}
    \caption{Similarity of output of many-body and GPE code for a $3\times3$
    lattice with no interactions.}
\end{figure}


We also appear to have gotten rid of the problems with symmetry that we
faced before (which were present even in the noninteracting case).

\begin{table}[H]
 \centering
 \begin{tabular}{c r r r}
  \multicolumn{4}{c}{Lattice site positions}\\
  \hline
  Column number &     Row 1     &     Row 2     &     Row 3\\
  \hline
   1            &     0.273     &     0.078     &   0.023  \\
   2            &     0.078     &     0.094     &   0.078  \\
   3            &     0.023     &     0.078     &   0.274  \\
   \hline
 \end{tabular}
 \caption{Symmetry in long term averages for a $3\times3$ system with $1$ 
 particle and $U_0=0$ with many-body code.}
\end{table}

The symmetry is also there in the GPE case.

\section{Interacting cases}

Initially, just looking at $3\times3$ lattices. More can be added as simulations 
finish.

For GPE, U=20 seems to agree with MB U=4 pretty well (they both keep the 
particles localised on the initial site). 

\begin{figure}[H]
   \centering
   \begin{gnuplot}[terminal=cairolatex, terminaloptions={lw 2}, scale=0.95]
       set yrange [0:1];
       set xrange [0:500];
       set xlabel "$\\displaystyle \\frac{t}{J}$"
       set ylabel "$\\langle n_{1} \\rangle$"
       plot  "./Data_for_comparisons/U20_RKF_3by3_T500.dat" u 1:2 w l lc 2 t ""        
    \end{gnuplot}
    \vspace*{-5mm}
    \caption{GPE simulation with U=20.}
\end{figure}

\begin{figure}[H]
   \centering
   \begin{gnuplot}[terminal=cairolatex, terminaloptions={lw 2}, scale=0.95]
       set yrange [0:5];
       set xrange [0:500];
       set xlabel "$\\displaystyle \\frac{t}{J}$"
       set ylabel "$\\langle n_{1} \\rangle$"
       plot  "./Data_for_comparisons/3by3_U4_T500_MB.dat" u 1:10 w l lc 1 t ""        
    \end{gnuplot}
    \vspace*{-5mm}
    \caption{Many-body simulation with U=4.}
\end{figure}



GPE U=0.1 disagrees quite dramatically with MB U=0.02, with the many body
version converging to a uniform spread but the GPE one staying very localised,
as it was previously.

\begin{figure}[H]
   \centering
   \begin{gnuplot}[terminal=cairolatex, terminaloptions={lw 2}, scale=0.95]
       set yrange [0:1];
       set xrange [0:500];
       set xlabel "$\\displaystyle \\frac{t}{J}$"
       set ylabel "$\\langle n_{1} \\rangle$"
       plot  "./Data_for_comparisons/U0.1_RKF_3by3_T500.dat" u 1:2 w l lc 2 t ""        
    \end{gnuplot}
    \vspace*{-5mm}
    \caption{GPE simulation with U=0.1.}
\end{figure}

\begin{figure}[H]
   \centering
   \begin{gnuplot}[terminal=cairolatex, terminaloptions={lw 2}, scale=0.95]
       set yrange [0:5];
       set xrange [0:500];
       set xlabel "$\\displaystyle \\frac{t}{J}$"
       set ylabel "$\\langle n_{1} \\rangle$"
       plot  "./Data_for_comparisons/3by3_U0.02_T500_MB.dat" u 1:10 w l lc 1 t ""        
    \end{gnuplot}
    \vspace*{-5mm}
    \caption{Many-body simulation with U=0.02.}
\end{figure}

Symmetry is also not a problem with these simulations (averages not shown here,
but they're always perfectly or almost-perfectly symmetric).

\section{Next steps}
- See what the properties we find for systems larger than 3x3 (especially 
for the weakly interacting GPE). 
\\
- Investigate the transition values of U again with the new code.
\\
- Once fixed Fourier Transform code is completed, look at comparing momentum 
distributions.


\end{document}
