\documentclass[a4paper,10pt]{article}
\usepackage[utf8]{inputenc}
\usepackage{amsmath}
\usepackage{braket}
%\usepackage{todonotes}
\usepackage{listings}
\usepackage{graphicx}
\usepackage{float}

\usepackage{hyperref}
\hypersetup{
    colorlinks=true,
    linkcolor=blue,
    filecolor=magenta,      
    urlcolor=blue,}
 
\urlstyle{same}

%opening
\title{Honours Research Project progress report}
\author{Nik Mitchell}


\begin{document}

\maketitle

\begin{abstract}

Systems of ultra-cold gases in light-induced periodic potentials (optical lattices) are of great experimental and theoretical interest, and have been since the first experimental 
realisation of Bose-Einstein condensation in 1995.
This is largely due to the parallels between the behaviour of BECs in an optical lattice and condensed matter systems, where 
electrons can be modelled as moving on a lattice generated by the periodic array of atom cores$^1$. 
\\\\
This comparison is useful because the systems of BECs can be finely tuned in ways that solids cannot. The lattice potential's shape, depth and spacing can be easily varied in optical lattices of BECs$^2$. 
Moreover, there are always impurities present in the solids we find in nature,
which can have significant impacts on the properties of the solid which cannot necessarily be accounted for by a perturbative approach which assumes these effects 
to be small. %reference this at some point from the book I meant to get from David

Optical lattices are highly uniform, so when dealing with BECs on optical lattices this problem of impurities does not arise.
\\\\
This project investigates the effects of changing the dimensionality of a lattice of ultracold bosons on the integrability of the system and its thermalisation behaviour. 
This will be done by starting with two (or more) one-dimensional chains and then gradually increasing the coupling
between sites on each chain.


\end{abstract}
\newpage
\section{First quantised representation of bosons on optical lattices}
There are a number of different contributions to the Hamiltonian for bosons on optical lattices, and these contributions can be seen to have analogues in condensed matter systems.
The individual bosons will have kinetic energy, and the potential created by the lattice also contributes to how the system evolves in time, so it must feature in the Hamiltonian.
The bosons may be interacting (we will assume this only occurs if they occupy the same lattice sites), and there may also be an external potential imposed that can vary in strength
from site to site. From this reasoning, we can construct a first quantised Hamiltonian for a 1D system of interacting bosons on an optical lattice, in the presence of an external potential $V_{ext}$
\begin{equation}
 \hat{H}=\sum_{i}-\frac{\hbar^{2}}{2m}  \partial_{x_{i}}^2+\sum_{j}V_{lattice}(R_{j})+V_{ext}(x)+\frac{U}{2}\sum_{i,j}\delta(x_{i}-x_{j}).
\end{equation}

Similar Hamiltonians are frequently used to describe systems of electrons on atomic lattices.
\\\\
%\todo{I'd like to have a diagram of the bosons on the optical lattice here}

\section{Many-particle wavefunctions and identical particles$^{3,4,5}$}
The Hamiltonian in the previous section would act on a bosonic many-particle wavefunction. A system of identical bosons must be symmetric under particle exchange, and this has 
implications for how we must write our many-particle wavefunctions. Let us first consider the case of two indistinguishable bosons. We have not yet determined the eigenstates 
of our Hamiltonian, but if we suppose we have the set of normalised single-particle wavefunctions $\ket{\lambda}$, and we have one boson in state $\ket{\lambda_1}$ and another
in state $\ket{\lambda_2}$, then we can write the two-particle wavefunction as

\begin{equation}
 \Psi(x_1,x_2)=\frac{1}{\sqrt{2}}\big(\braket{x_1|\lambda_1}\braket{x_2|\lambda_2}+\braket{x_1|\lambda_2}\braket{x_2|\lambda_1}\big),
\end{equation}
or in Dirac bra-ket notation, the two-body states would be represented as
\begin{equation}
 \ket{\lambda_1,\lambda_2}=\frac{1}{\sqrt{2}}\big(\ket{\lambda_1}\otimes \ket{\lambda_2}+\ket{\lambda_2}\otimes\ket{\lambda_1}\big)
\end{equation}

The number of permutations that one must account for grows extremely quickly as particle number increases. A properly symmetrised and normalised N-body state can be
represented as 
\begin{equation}
 \ket{\lambda_1,\lambda_2,...,\lambda_N}=\frac{1}{\sqrt{N!\prod_{\lambda=0}^{\infty}{(n_{\lambda})}}}\sum_{\mathcal{P}}\lambda_{\mathcal{P}1}\otimes\lambda_{\mathcal{P}2}\otimes...\otimes\lambda_{\mathcal{P}N}
\end{equation}
where $n_{\lambda}$ is the number of particles in state $\lambda$, and the summation runs over all $N!$ permutations of the set of quantum numbers $\{ \lambda_1,\dots,\lambda_N\}$.
\\\\
This formalism has a number of shortcomings. The most important of these for this project is that it is extremely cumbersome for practical computation because of the large number of entities that need to be 
represented. To avoid this, we shall adopt the ``second quantisation'' formalism, which is much better suited to dealing concisely with large numbers of indistinguishable particles.

\section{Second Quantisation and the Occupation Number Representation}
\subsection{The Occupation Number Representation}
The formalism that we have hitherto discussed explicitly represents a significant amount of redundant information, in the sense that it deals separately with the scenarios ``particle 1 in state $\lambda_1$ and 
particle 2 in state $\lambda_2$'' and ``particle 2 in state $\lambda_1$ and particle 1 in state $\lambda_2$''. Taking into account the indistinguishability of the particles, it is clear that these two scenarios
are precisely identical. A more efficient approach consists of describing the number of particles in a particular state $\lambda_i$, i.e. using the occupation number representation. When doing this, a general 
state can be written as a linear superposition
\begin{equation}
 \ket{\Psi}=\sum_{n_1,n_2,...}c_{n1,n2,...}\ket{n_1,n_2,...}
\end{equation}

In the scenario which this project will be working with, the occupation numbers refer to the number of bosons on a particular site of the lattice.
Having established this, we can define creation and annihilation operators that add and subtract particles from eigenstates, that is

\begin{equation}
 \hat{a}_{j}^{\dagger}\ket{n_1,n_2,\dots,n_{j},\dots}=\sqrt{n_j+1}\ket{n_1,n_2,\dots,n_{j}+1,\dots}
\end{equation}

and 

\begin{equation}
 \hat{a}_{j}\ket{n_1,n_2,\dots,n_{j},\dots}=\sqrt{n_j}\ket{n_1,n_2,\dots,n_{j}-1,\dots}
\end{equation}
These operators have very important commutation relations
\begin{equation}
\begin{align*}
 [\hat{a}_{j}^{\dagger},\hat{a}_{k}^{\dagger}]=0, \ \ \ \ \ [\hat{a}_{j},\hat{a}_{k}]=0,\ \ \ \ [\hat{a}_{j},\hat{a}_{k}^{\dagger}]=\delta_{jk}.
 \end{align*}
\end{equation}

We can also define the number operator $\hat{n}_j=\hat{a}_{j}^{\dagger}\hat{a}_j$ with the property that
\begin{equation}
 \hat{n}_{j}\ket{n_1,n_2,\dots,n_{j},\dots}=n_{j}\ket{n_1,n_2,\dots,n_{j},\dots}
\end{equation}

The occupation number eigenstates form the basis of the N-particle Hilbert space that we are working in (the Fock space $\mathcal{F}^N$), and it 
is useful to observe that any occupation number eigenstate can be created from the empty (or ``vacuum'') state $\ket{0}$ by repeated action of 
creation operators
\begin{equation}
 \ket{n_1,n_2,\dots}=\prod_i\frac{1}{\sqrt{n_i!}}(\hat{a}_i^{\dagger})^{n_i}\ket{0}
\end{equation}
The Fock space can be built up from N single-particle Hilbert spaces $\mathcal{H}$.
\begin{equation}
 \mathcal{H}^N=\mathcal{H}\otimes \mathcal{H} \otimes \dots\otimes \mathcal{H}
\end{equation}

The Fock space $\mathcal{F}^N$ is a subspace of $\mathcal{H}^N$ subject to certain symmetry constraints.

Now that we have established the space and states that we are operating within, we now need to rewrite the Hamiltonian in such a form that it can operate on number eigenstates (i.e. we 
``second quantise'' it).

\subsection{Second Quantisation of the Hamiltonian}
We can second quantise the Hamiltonian initially used to describe our system through the use of boson field operators $\hat{\psi}^{\dagger}(x)$ and $\hat{\psi}(x)$, which create and destroy 
particles at particular spatial locations (we will define these more explicitly later). The first two terms can be transformed to %\todo{is transformed the correct term?}
\begin{equation}
 \int  \hat{\psi}^{\dagger}(x) \bigg(  \sum_{i}-\frac{\hbar^{2}}{2m}  \partial_{x_{i}}^2+\sum_{j}V_{lattice}(R_{j})  \bigg)    \hat{\psi}(x)dx
\end{equation}

In a weak lattice, Bloch's theorem (in 1D) tells us that the eigenstates of a particle in a periodic potential has the form
\begin{equation}
 \phi_q(x)=e^{iqx}u_{q}(x)
\end{equation}
where $q$ is the quasi-momentum and $u_q(x)$ is periodic with the same period as the lattice,$d$. We will be considering scenarios in which the strength of the lattice potential
is such that the bosons are not completely localised, but such that the overlap between the wavefunctions of particles on particular lattice sites have effectively zero overlap 
with non-nearest neighbours. Under these conditions, the wavefunctions can be described by the localised Wannier functions
\begin{equation}
\psi(R;r)=\frac{1}{d}\int dq e^{iRq}\phi_q(x)
\end{equation}
which are superpositions of Bloch functions. We can use this to rewrite the field operators as 
\begin{equation}
 \hat{\psi}(x)=\sum_j \hat{a}_{j}(t)\psi(R_j-x)
\end{equation}
With this in mind, we can write
\begin{equation}
 \int  \sum_j\hat{a}_j^{\dagger}\psi^{*}(R_j-x) \bigg(  \sum_{i}-\frac{\hbar^{2}}{2m}  \partial_{x_{i}}^2+\sum_{j}V_{lattice}(R_{j})  \bigg)  \sum_l  \hat{a}_l\psi(R_l-x)dx=\sum_{j,l} J_{j,l}\hat{a}_{j}^{\dagger}\hat{a}_l
\end{equation}
Where
\begin{equation}
 J_{j,l}=\int  \psi^{*}(R_j-x) \bigg(  \sum_{i}-\frac{\hbar^{2}}{2m}  \partial_{x_{i}}^2+\sum_{j}V_{lattice}(R_{j}))  \bigg)  \psi(R_l-x)dx
\end{equation}
characterises the strength of the hopping between sites $l$ and $j$, which (intuitively) depends on the combination of the lattice depth and the kinetic energy. We take this this hopping strength to be negligible
for sites which are not nearest-neighbours. Of the permitted hopping interactions, we assume that each has identical strength. This allows us to write
\begin{equation}
 \int  \hat{\psi}^{\dagger}(x) \bigg(  \sum_{i}-\frac{\hbar^{2}}{2m}  \partial_{x_{i}}^2+\sum_{j}V_{lattice}(R_{j})  \bigg)    \hat{\psi}(x)dx=J\sum_{i}(\hat{a}^\dagger_{i}\hat{a}_{i+1}+c.c.).
\end{equation}

We can go through a similar process for the contributions of the external potential and on-site interactions between bosons. The external potential gives an on-site energy that can vary
at different locations in the lattice

\begin{equation}
 \int  \hat{\psi}^{\dagger}(x) V_{ext}(x)  \hat{\psi}(x)dx = \sum_i \epsilon_i \hat{a}_i^{\dagger}\hat{a}_i.
\end{equation}

Second quantising the term for on-site interaction between bosons (which is a two-particle operator) requires integration between two sets of boson field operators

\begin{equation}
 \int  \hat{\psi}^{\dagger}(x)\hat{\psi}^{\dagger}(x) \frac{U}{2}\sum_{i,j}\delta(x_{i}-x_{j})  \hat{\psi}(x) \hat{\psi}(x) dx = \frac{U}{2}\sum_i \hat{a}_i^{\dagger}\hat{a}_i^{\dagger}\hat{a}_i\hat{a}_i
\end{equation}


Putting these together, we arrive at the Bose-Hubbard model described in the next section.
\newpage




\section{The Bose-Hubbard model}

The Hamiltonian for a weakly interacting BEC in an 1-dimensional optical lattice and subject to harmonic trapping potential is given by

\begin{equation}
 \hat{H}=J\sum_{i}(\hat{a}^\dagger_{i}\hat{a}_{i+1}+c.c.)+\frac{U}{2}\sum_{i}\hat{a}^\dagger_{i}\hat{a}^\dagger_{i}\hat{a}_{i}\hat{a}_{i}+\sum_{i}{\epsilon_i}\hat{a}^\dagger_{j}\hat{a}_{j},
\end{equation}
The $\epsilon_i$'s refer to on-site energies at each lattice site (influenced by the harmonic trap), and the middle term gives an interaction energy when there is more than one particle 
on a particular site.
\\\\
This project will look at scenarios where there is no external harmonic potential which produces different on-site energies for different sites, and will also 
assume that interactions between bosons are negliglible. Under these conditions, the Hamiltonian in one dimension reduces to

\begin{equation}
\begin{align*}
\hat{H}=&J\sum_{i}\hat{a}^\dagger_{i}\hat{a}_{i+1}+c.c.\ .
\end{align*}
\end{equation}

The atoms in the lattice are still subject to the lattice potential. We can see this from the definition of $J$ as the ``hopping integral'' in terms of the Hamiltonian of
the system in the wavefunction representation.
\\\\
 The main aim of this project is to analyse the predicted behaviour of this system as it transitions from 1 to 2 dimensions. This transition will be implemented by initially 
 creating two one-dimensonal chains which are each described by the above Hamiltonian, and then introducing coupling between a site on the first chain and a site on the second
 chain, which would allow for inter-chain hopping. The Hamiltonian used to describe this system is
 
\begin{equation}
\hat{H}=(J\sum_{i,j}\hat{a}^\dagger_{i,j+1}\hat{a}_{i,j} + J'\hat{a}^\dagger_{i,b}\hat{a}_{i+1,b})+c.c.
\end{equation}
where the $i$ index denotes which chain is being referred to, the $j$ index denotes how far along the chain a site is, and inter-chain hopping is allowed between the $b^{th}$ 
sites of each chain. The value of $J'$ will be increased incrementally from zero to $J$, in doing so taking the system from quasi-1D to fully 2-dimensional. We will look at how
this change impacts the revival behaviour of the expectation value of the annihilation operator.

\section{Coherent states$^6$}

We can use these number eigenstates to build coherent states $\ket{\alpha}$, which are defined as the eigenstates of the annihilation operator (with eigenvalue $\alpha$)
\begin{equation}
\ket{\alpha}=e^{-\frac{|\alpha|^2}{2}} \sum_{n=0}^{\infty} \frac{\alpha^n}{\sqrt{n!}}\ket{n}.  
\end{equation}

If the system is in one of these coherent states, the expectation value of the annihilation operator will be $\alpha$, which we can choose to be $1$. %\todo{I'm no longer sure we can do this}. 
It has been observed that in 1 dimension, this expectation value drops initially as the state is evolved in time, but eventually exhibits a revival back to unity$^7$. 
This is what one would anticipate inevitably happening, since the time evolution of an isolated quantum system is linear$^8$.
\\\\
We set up a 1-dimensional chain of lattice sites, and look for this revival phenomenon. Then we incrementally couple it to a second chain whilst monitoring the revival amplitude and the characteristic time 
of revival. We anticipate that these quasi-2D systems will exibit revival, but with decreasing amplitude and/or increasing time between instances of revival. We are also interested in the implications of this 
breaking of integrability on how well the final states to which the system thermalises can be predicted by statistical mechanics. 

\section{Integrable systems}
An integrable system is one for which the number of independent constraints is equal to the number of degrees of freedom. Since the system we are dealing with
is closed, it is subject to the constraints of constant energy and particle number. For a single atom in 1 dimension, its degrees of freedom are its 1D position and momentum.
So we expect this system to be completely integrable. Once more particles are added, or the system is extended to two dimensions, then if we do not introduce a commensurate number
of new constraints, integrability will be broken.


\section{Experimental studies}
\subsection{A Quantum Newton's Cradle$^9$}

This study investigated the thermalisation of an out-of-equilibrium 1D Bose gases, which are nearly-integrable systems. The authors prepared
of out-of-equilibrium arrays of trapped one-dimensional (1D) Bose gases, each containing from $40$ to $250$ $^{87}$Rb atoms, and found them to not noticeably equilibrate,
even after thousands of collisions. 

These observations extended from the Tonks–Girardeau regime, which has very strong repulsive interactions between bosons so only pairwise collisions can occur, to the intermediate 
coupling regime, where there can be three- (or more) body collisions. 

\subsection{Relaxation in a Completely Integrable Many-Body Quantum System$^8$}

Inspired by the ``A Quantum Newton's Cradle'' study, investigations were made into a completely integrable many-body quantum system. In this experiment, hard-core bosons (which cannot
occupy the same state and so behave like fermions, but without exchange antisymmetry) on a 1D lattice were used. The authors found that the system can undergo relaxation to an equilibrium 
state. The properties of the state that the system relaxed to were not given by the grand-canonical ensemble, but rather by a generalised Gibbs ensemble, in which the partition function 
is extended to include all of the integrals of motion. They further showed that their generalized equilibrium state carries more memory of the initial conditions than the usual thermodynamic one.


\section{Preliminary results}
The QuTiP 3.0 package$^10$ for Python has been used extensively in the simulations that have been run, both for creating the state vectors and evolving them in time. 
So far, only situations in which one particle is placed on a given site have been considered. It has been found that the expectation value of the number operator for the site that 
the particle is initially placed on tends to a particular value (although oscillations around this do not disappear when the simulation is run much longer). This value does not change
based on the relative values of $J$ and $J'$. However, the relative values of $J$ and $J'$ do influence how long it takes for this behaviour to set in. A characteristic time was 
defined to quantify this difference. This was done by iterating backwards through the timesteps until a time at which the absolute difference between the mean of the oscillations in
expectation value and the expectation value at that particular time exceeds a chosen value. 

\begin{figure}[H]
 \includegraphics[width=8cm]{scale_factors_and_characteristic_times}
 \centering
\end{figure}

It was found that a weaker inter-chain coupling corresponded with delayed onset of long-term behaviour.

\section*{References}
\begin{enumerate}
 \item I. Bloch, J. Dalibard, S. Nascimbène, \textit{Quantum simulations with ultracold quantum gases}, Nature \textbf{8}, 267-276 (2012)\\
 \item I. Morsch, M. Oberthalerm, \textit{Dynamics of Bose-Einstein Condensates in optical lattices}, Reviews of Modern Physics \textbf{78}, 180-215 (2006)\\
 \item J. Negele, H. Orland, \textit{Quantum Many-Particle Systems}, Westiew Press (2008)\\
 \item J. Inkson, \textit{Many-body Theory of Solids}, Plenum Press (1984) \\
 \item A. Atland, B Simons, \textit{Condensed Matter Field Theory}, Cambridge University Press  (2010)\\
 \item A Furushawa - \textit{Quantum States of Light}, Springer (2015)\\
 \item A. Polkovnikov, \textit{Phase space representation of quantum dynamics}, Annals of Phys. \textbf{325}, 1790 (2010)\\
 \item M. Rigol et al., \textit{Thermalization and its mechanism for generic isolated quantum systems}, Physical Review Letters \textbf{98}, 050405 (2007)\\
 \item T. Kinoshita, T. Wenger, D. Weiss, \textit{A Quantum Newton's Cradle}, Nature \textbf{440}, 900-903 (2006)\\
 \item J. Johansson, P. Nation, F. Nori, \textit{QuTiP: An open-source Python framework for the dynamics of open quantum systems}, Comp. Phys. Comm. \textbf{183}, 1760–1772 (2012)\\
\end{enumerate}

%Note: \url{https://arxiv.org/pdf/1007.5331.pdf} has a good description of what quantum ergodicity is on page 10
%\\ Note: \url{http://www.physicspages.com/tag/field-operators/} was the thing that made field operators make some sense for me
%\\Note: \url{http://www.nucleares.unam.mx/~alberto/apuntes/altland.pdf} was super useful for background second quantisation.
%\\Note: \url{http://iopscience.iop.org/article/10.1209/epl/i2004-10265-7/pdf} mentions finding the 1D B-H model to not be integrable. This could be v useful when including U(n*n-n) term
%\\Note: \url{https://arxiv.org/pdf/cond-mat/0410614v1.pdf} might have good stuff for background section of dissertation
%\newpage


%\section{Additional questions for Danny}

%After what I've already written, I want to start going into looking at how much of the phase space can be explored. The term ''phase space'' is confusing to me, because 
%I've also encountered the term ``constant-energy manifold''. If you start in a particular state with a certain energy, then for a closed system I don't see how you'll ever get to parts of the space with a different energy
%to your initial one without breaking conservation of energy. What am I missing here?
%\\\\
%Why does \url{https://www.nature.com/nature/journal/v419/n6902/pdf/nature00968.pdf} not have a tunneling term? It talks a lot about revival and coherent states. Is it relevant regardless?
%\\\\
%What, if any, is the difference between ``tunneling'' and ``hopping''?
%\\\\



\end{document}